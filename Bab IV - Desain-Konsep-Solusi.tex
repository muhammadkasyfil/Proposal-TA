\chapter{DESAIN KONSEP SOLUSI}

\section{Diagram Konseptual Sistem}
Bab ini menyajikan desain konseptual dari solusi yang diusulkan melalui perbandingan alur sistem sebelum (\textit{before}) dan sesudah (\textit{after}) pengembangan prototipe \textit{dropper spyware} mode \textit{stealth}. Diagram konseptual ini menunjukkan perbedaan fundamental antara kondisi keamanan siber saat ini dan mekanisme eksekusi \textit{dropper} dengan teknik pengelakan (\textit{evasion techniques}) terintegrasi yang akan dievaluasi.

\subsection{Sistem Sebelum (Before)}
Sistem deteksi \textit{anti-malware} saat ini beroperasi dengan asumsi bahwa \textit{malware} akan meninggalkan jejak yang terdeteksi melalui \textit{signature matching} atau \textit{behavioral anomalies}. Pertama, ketika pengguna menjalankan file executable, \textit{antivirus} melakukan \textit{on-access scan} dengan membandingkan \textit{hash} file terhadap basis data \textit{malware} yang diketahui. Jika \textit{hash} cocok, file diblokir dan pengguna diberi notifikasi. Jika \textit{hash} tidak cocok (file \textit{unknown}), sistem melanjutkan dengan \textit{behavioral monitoring} pada tingkat file system dan API calls.

Selama \textit{behavioral monitoring}, sistem memantau indikator-indikator mencurigakan seperti penulisan file ke \textit{system directories}, perubahan \textit{registry}, atau \textit{network connections} yang abnormal. Jika aktivitas mencurigakan terdeteksi, sistem mengeluarkan peringatan kepada pengguna. Namun, jika tidak ada \textit{behavioral anomalies} yang terdeteksi—misalnya karena \textit{dropper} menggunakan \textit{fileless execution}, operasi \textit{in-memory}, atau \textit{anti-analysis techniques}—maka \textit{malware} dapat beroperasi tanpa terdeteksi. Diagram alur sistem \textit{before} disajikan pada Gambar~\ref{fig:before}.

\begin{figure}[h]
  \centering
  \includegraphics[width=0.7\textwidth]{image/before.png}
  \caption{Alur Deteksi Anti-Malware Sebelum (Before)}
  \label{fig:before}
\end{figure}

\subsection{Sistem Sesudah (After)}
Sistem sesudah menggambarkan mekanisme eksekusi prototipe \textit{dropper spyware} mode \textit{stealth} dalam lingkungan lab terisolasi dengan instrumentasi dan \textit{monitoring} penuh. Alur dimulai ketika peneliti men-\textit{deploy binary dropper} ke VM target di \textit{lab environment}. Setelah dieksekusi, \textit{dropper} melakukan inisialisasi dan \textit{self-verification} untuk memastikan biner tidak \textit{corrupt}.

Berikutnya, \textit{dropper} melakukan \textit{unpacking} atau \textit{deobfuscation} terhadap bagian-bagian kode yang disamarkan sehingga logika inti dapat dijalankan di memori. Tahap krusial selanjutnya adalah \textit{environment checking}, di mana \textit{dropper} memeriksa keberadaan VM hypervisor, \textit{sandbox tools}, \textit{debugger}, atau \textit{analysis tools}. Jika lingkungan terdeteksi tidak sesuai (VM/sandbox/\textit{debugger}), \textit{dropper} melakukan \textit{graceful exit} tanpa meninggalkan artefak yang mudah diamati sehingga sistem pemantau tidak melihat eksekusi \textit{payload} yang mencurigakan.

Jika lingkungan dianggap aman, \textit{dropper} memuat dan menyiapkan \textit{payload} yang sebelumnya telah disamarkan atau dienkripsi menggunakan fungsi kriptografi dari library Rust. \textit{Payload} kemudian dieksekusi secara \textit{in-memory}, misalnya dengan menyuntikkan kode ke proses sah yang sudah berjalan. Setelah \textit{payload} berjalan, \textit{dropper} melakukan \textit{cleanup} menyeluruh untuk menghapus dirinya dari memori dan tidak meninggalkan mekanisme \textit{persistence}. Diagram alur sistem \textit{after} disajikan pada Gambar~\ref{fig:after}.

\begin{figure}[h]
  \centering
  \includegraphics[width=0.7\textwidth]{image/after.png}
  \caption{Alur Eksekusi Dropper Spyware Mode Stealth (After)}
  \label{fig:after}
\end{figure}

\subsection{Perbandingan Sistem Before dan After}
Perbandingan antara sistem \textit{before} dan \textit{after} menunjukkan perbedaan fundamental dalam pendekatan deteksi versus eksekusi \textit{malware} dengan \textit{evasion techniques} terintegrasi. Tabel~\ref{tab:perbandingan-before-after} merangkum dimensi-dimensi kunci perbandingan.

\begin{longtable}{p{3.0cm}p{5.2cm}p{6.0cm}}
  \caption{Perbandingan Sistem Sebelum dan Sesudah Pengembangan\label{tab:perbandingan-before-after}}\\
  \toprule
  \textbf{Aspek} & \textbf{Sistem Before} & \textbf{Sistem After} \\
  \midrule
  \endfirsthead
  \toprule
  \textbf{Aspek} & \textbf{Sistem Before} & \textbf{Sistem After} \\
  \midrule
  \endhead
  \midrule
  \multicolumn{3}{r}{\textit{bersambung}} \\
  \endfoot
  \bottomrule
  \endlastfoot
  Fokus Deteksi & \textit{Signature-based} dan \textit{behavioral anomalies} pada disk atau \textit{registry} & Alur eksekusi ter-\textit{instrumentasi} dengan visibilitas penuh dari inisialisasi hingga \textit{payload execution} \\
  Jenis Malware Target & \textit{Malware} generik dengan \textit{evasion} minimal & \textit{Dropper spyware} mode \textit{stealth} dengan beberapa teknik pengelakan terintegrasi \\
  Environment & Sistem produksi dengan aktivitas pengguna nyata & Lab terisolasi dengan kondisi terkontrol dan terukur \\
  Blind Spots & \textit{Fileless execution}, operasi \textit{in-memory}, \textit{environment checking}, \textit{behavioral stealth} & Minimal; setiap tahap dipantau (Sysmon, \textit{anti-malware logs}, \textit{network capture}, analisis memori) \\
  Visibilitas Peneliti & Terbatas pada \textit{alert deteksi} & Komprehensif; tiap tahap eksekusi dapat dihubungkan dengan respons tiap produk \textit{anti-malware}/EDR \\
  Output Evaluasi & Hanya ``terdeteksi/tidak terdeteksi'' & Metrik terkuantifikasi (\textit{Detection Rate}, \textit{Evasion Rate}, \textit{Time-to-Detect}, artefak forensik per tahap) \\
\end{longtable}

Tabel IV.1 menunjukkan bahwa sistem \textit{after} menyediakan lingkungan penelitian yang terstruktur dan terukur untuk mengevaluasi efektivitas \textit{evasion techniques} \textit{dropper} serta mengidentifikasi \textit{gap} spesifik pada kemampuan deteksi sistem \textit{anti-malware} dan EDR yang ada.

\section{Penjelasan Desain Solusi}
Penjelasan desain solusi menjabarkan rancangan dari setiap modul \textit{dropper} dan bagaimana integrasi modul tersebut menghasilkan mekanisme eksekusi yang \textit{stealth} dan sulit dideteksi.

\subsection{Desain Inisialisasi dan Verifikasi}
Pada tahap inisialisasi, \textit{dropper} melakukan \textit{internal integrity checking} untuk memastikan biner tidak \textit{corrupt} atau ter-\textit{tamper}. Verifikasi dapat dilakukan menggunakan \textit{checksum} atau \textit{keyed hashing} pada bagian-bagian kritis biner. Jika pemeriksaan gagal, \textit{dropper} segera berhenti tanpa operasi lanjutan. Jika lolos, \textit{dropper} melanjutkan ke proses \textit{unpacking} terhadap bagian kode yang disamarkan sehingga logika inti dapat dijalankan di memori tanpa menulis file baru ke disk.

\subsection{Desain Anti-Analysis (Environment Checking)}
Modul \textit{environment checker} melakukan pemeriksaan sistematis terhadap karakteristik lingkungan untuk mengidentifikasi \textit{analysis environment}. Pemeriksaan dapat mencakup:
\begin{enumerate}
  \item Pencarian indikasi hypervisor/VM.
  \item Pengecekan \textit{registry} untuk artefak khas perangkat lunak virtualisasi.
  \item Enumerasi proses untuk mendeteksi \textit{analysis tools} populer.
  \item Pemeriksaan tanda-tanda proses sedang di-\textit{debug}.
\end{enumerate}
Jika ditemukan indikasi kuat bahwa sistem sedang dianalisis, \textit{dropper} melakukan \textit{controlled exit} tanpa mengeksekusi \textit{payload} dan tanpa memicu \textit{error} yang mencolok.

\subsection{Desain In-Memory Execution (Fileless Loading)}
Modul \textit{Loader \& In-Memory Executor} bertanggung jawab memuat \textit{payload} ke memori dan menjalankannya tanpa membuat file baru di disk. \textit{Payload} disimpan dalam biner \textit{dropper} dalam bentuk yang telah disamarkan atau dienkripsi sederhana menggunakan fungsi kriptografi dari library Rust. Saat kondisi lingkungan dinyatakan aman, modul ini:
\begin{enumerate}
  \item Mengambil \textit{payload} tersemat dan mengembalikannya ke bentuk siap eksekusi (misalnya melalui fungsi \textit{decrypt} atau \textit{decode} dari library).
  \item Menempatkan \textit{payload} ke wilayah memori yang dialokasikan secara dinamis.
  \item Menjalankan \textit{payload}, misalnya dengan menyuntikkan eksekusi ke proses sah yang sudah ada.
\end{enumerate}
Seluruh langkah dilakukan di memori untuk meminimalkan artefak forensik dan mengurangi peluang terdeteksi oleh mekanisme \textit{file-based detection}.

\subsection{Desain Komponen Modul Dropper}
Integrasi dari semua modul dirancang untuk menciptakan alur eksekusi yang kohesif dan \textit{seamless}. Tabel~\ref{tab:komponen-dropper} merangkum komponen-komponen modul utama dan fungsi evasive-nya.

\begin{longtable}{p{0.8cm}p{4.0cm}p{5.8cm}p{3.6cm}}
  \caption{Komponen Modul Dropper dan Fungsi Evasion\label{tab:komponen-dropper}}\\
  \toprule
  \textbf{No} & \textbf{Modul} & \textbf{Fungsi Utama} & \textbf{Teknik Evasion} \\
  \midrule
  \endfirsthead
  \toprule
  \textbf{No} & \textbf{Modul} & \textbf{Fungsi Utama} & \textbf{Teknik Evasion} \\
  \midrule
  \endhead
  \midrule
  \multicolumn{4}{r}{\textit{bersambung}} \\
  \endfoot
  \bottomrule
  \endlastfoot
  1 & Initialization dan Self-Verify & Memastikan biner tidak \textit{corrupt}/ter-\textit{tamper} & \textit{Code integrity verification} \\
  2 & Code Unpacking/Deobfuscation & Mengeluarkan logika inti dari kode yang disamarkan di \textit{runtime} & \textit{Obfuscation bypass}, \textit{unpacking in-memory} \\
  3 & Environment Checker & Memeriksa indikasi VM, \textit{sandbox}, \textit{debugger}, dan \textit{analysis tools} & \textit{Anti-VM}, \textit{anti-sandbox}, \textit{anti-debug} \\
  4 & Payload Preparation \& Validation & Mengambil \textit{payload} tersemat dan menyiapkannya untuk dieksekusi & \textit{Payload obfuscation}, \textit{simple encryption/encoding} \\
  5 & In-Memory Loader & Mengalokasikan memori dan mengeksekusi \textit{payload} tanpa file di disk & \textit{Fileless execution}, \textit{injection} ke proses sah \\
  6 & Cleanup dan Self-Termination & Menghapus jejak eksekusi \textit{dropper} dan menutup \textit{resource} yang digunakan & \textit{Non-persistence}, \textit{anti-forensics} \\
\end{longtable}

Setiap modul dirancang saling melengkapi sehingga membentuk alur eksekusi berlapis yang meningkatkan probabilitas keberhasilan pengelakan terhadap berbagai lapisan deteksi.

\section{Rancangan Lingkungan Pengujian}
Lingkungan pengujian dirancang untuk mensimulasikan kondisi \textit{execution dropper spyware mode stealth} dalam konteks terisolasi yang memungkinkan peneliti untuk mengobservasi dan merekam setiap tahap eksekusi dengan \textit{full visibility}, sekaligus memastikan keamanan dan \textit{isolation} dari jaringan eksternal. Lingkungan pengujian dibangun di atas \textit{single host machine} (laptop pribadi peneliti) dengan menggunakan virtualisasi untuk menciptakan \textit{multiple isolated guest VMs} yang masing-masing dikonfigurasi dengan produk \textit{anti-malware} atau EDR berbeda.

\textit{Isolasi network} dilakukan menggunakan \textit{virtual networking} dengan konfigurasi \textit{host-only} atau NAT, memastikan bahwa guest VMs tidak memiliki akses ke jaringan internet eksternal. Semua komunikasi C2 (\textit{Command and Control}) dimock di dalam \textit{lab network} menggunakan C2 server \textit{dummy} lokal untuk merepresentasikan komunikasi \textit{malicious} tanpa risiko \textit{escape} ke internet. Monitoring pada \textit{multiple layers} dilakukan untuk mengumpulkan data komprehensif tentang perilaku \textit{dropper} dengan cara sebagai berikut.
\begin{itemize}
  \item \textbf{File system layer} menggunakan Sysmon untuk merekam semua \textit{file operations}.
  \item \textbf{Registry layer} menggunakan WMI atau \textit{custom monitoring scripts} untuk merekam \textit{registry changes}.
  \item \textbf{Process layer} menggunakan \textit{API hooking} atau Sysmon untuk merekam \textit{process creation}, \textit{thread creation}, dan \textit{module loading}.
  \item \textbf{Network layer} menggunakan \textit{packet capture tools} (\textit{tcpdump} atau Wireshark) untuk merekam semua \textit{network communications}.
  \item \textbf{Memory layer} menggunakan \textit{kernel debugger} atau \textit{live memory analysis tools} untuk \textit{examining memory contents} dan \textit{process memory maps}.
\end{itemize}

\section{Spesifikasi Mesin Uji}
Pengujian \textit{dropper} dilakukan menggunakan infrastruktur lokal yang terdiri dari satu \textit{host physical} (laptop pribadi peneliti) dan beberapa guest VMs yang dijalankan di atasnya. Spesifikasi \textit{hardware} dan \textit{software} untuk mesin uji disajikan pada Tabel~\ref{tab:spesifikasi-uji}.

\begin{longtable}{p{3.0cm}p{5.8cm}p{5.8cm}}
  \caption{Spesifikasi Lingkungan Uji\label{tab:spesifikasi-uji}}\\
  \toprule
  \textbf{Komponen} & \textbf{Spesifikasi} & \textbf{Keterangan} \\
  \midrule
  \endfirsthead
  \toprule
  \textbf{Komponen} & \textbf{Spesifikasi} & \textbf{Keterangan} \\
  \midrule
  \endhead
  \midrule
  \multicolumn{3}{r}{\textit{bersambung}} \\
  \endfoot
  \bottomrule
  \endlastfoot
  \multicolumn{3}{l}{\textbf{Host Physical (Laptop Pribadi)}} \\
  Processor & Intel Core i5 atau i7 atau AMD Ryzen 5 atau 7 (minimum 4 \textit{cores}) & Untuk menjalankan \textit{multiple VMs} secara bersamaan dengan performa \textit{adequate} \\
  RAM & 16 GB minimum & Alokasi 4–6 GB per guest VM, tersisa untuk host OS dan \textit{monitoring tools} \\
  Storage & SSD 512 GB atau lebih & Untuk host OS, Rust \textit{toolchain}, VirtualBox atau VMware, guest VM images, dan \textit{logging data} \\
  OS Host & Windows 10 Pro atau 21H2 atau lebih atau Windows 11 & Sebagai host OS untuk \textit{hypervisor} dan \textit{development environment} \\
  Hypervisor & VirtualBox 7.0 atau lebih atau VMware Workstation & Untuk membuat dan menjalankan guest VMs terisolasi \\
  \midrule
  \multicolumn{3}{l}{\textbf{Guest VM 1-3}} \\
  OS & Windows 10 Pro (Build 19042 atau lebih) atau Windows 11 (Build 22621 atau lebih) & \textit{Representative} terhadap \textit{enterprise desktop} yang menjadi target \textit{dropper} \\
  vCPU & 2 \textit{cores} & \textit{Sufficient} untuk menjalankan OS dan \textit{single malware execution} \\
  vRAM & 4-6 GB & \textit{Adequate} untuk Windows 10 atau 11 dan \textit{anti-malware software} \\
  Storage & 40-50 GB \textit{virtual disk} & \textit{Space} untuk OS, \textit{anti-malware}, Sysmon, \textit{logging}, dan \textit{temporary test files} \\
  Network & \textit{Host-only virtual switch} & \textit{Isolated network} tanpa akses ke jaringan eksternal; semua \textit{traffic} terikat di \textit{lab network} \\
  Monitoring Tools & Sysmon, Process Explorer, Registry Editor & Untuk merekam dan mengobservasi aktivitas sistem selama \textit{dropper execution} \\
  \midrule
  \multicolumn{3}{l}{\textbf{Anti-Malware atau EDR Products}} \\
  VM 1 & Windows Defender (\textit{built-in}) & \textit{Baseline anti-malware} dengan \textit{behavioral monitoring} \\
  VM 2 & AVG Free atau Kaspersky Free & \textit{Alternative anti-malware} dengan \textit{heuristics} dan \textit{cloud-based detection} \\
  VM 3 & McAfee Personal Security atau Norton LifeLock Trial & \textit{Commercial-grade anti-malware} dengan \textit{advanced behavioral analysis} \\
  \midrule
  \multicolumn{3}{l}{\textbf{Monitoring dan Logging Infrastructure}} \\
  Sysmon Agent & \textit{Deployed} di semua guest VMs & Merekam \textit{process creation}, \textit{file writes}, \textit{registry modifications}, \textit{network connections} pada \textit{Event ID} level dengan \textit{custom rule sets} \\
  Network Monitoring & \textit{Packet capture tools} (\textit{tcpdump} atau Wireshark) & \textit{Running} pada host OS atau \textit{dedicated monitoring VM} untuk \textit{capturing C2 dummy communication} dan \textit{anomalous traffic} \\
  Log Aggregation & \textit{Text files} atau \textit{simple SQLite database} & Kumpulkan \textit{logs} dari Sysmon, \textit{anti-malware}, dan \textit{monitoring tools} ke \textit{centralized location} di host untuk \textit{post-analysis} \\
  Analysis Tools & x64dbg, WinDbg, \textit{strings}, IDA Free & Untuk \textit{debugging dropper execution}, \textit{memory analysis}, dan \textit{binary inspection} di \textit{post-execution} fase \\
\end{longtable}

Spesifikasi ini dirancang untuk \textit{balanced} antara \textit{resource constraints} dari \textit{single laptop} pribadi dan kebutuhan untuk \textit{adequate monitoring} dan \textit{evaluation} dari \textit{multiple anti-malware products} secara \textit{concurrent}. Pengujian dapat dilakukan secara \textit{sequential} jika \textit{resource} terbatas, atau \textit{parallel} jika host memiliki \textit{resource} yang cukup.
